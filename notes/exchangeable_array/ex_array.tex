% !TEX TS-program = pdflatexmk
\documentclass{article}

\usepackage{graphicx}
\usepackage{amssymb}
\usepackage{amsmath}
\usepackage[a4paper, total={6in, 8in}]{geometry}
\usepackage{amsthm}


\newtheorem{theorem}{Theorem} %This is the example presented in the introduction but it has the additional parameter [section] that restarts the theorem counter at every new section.
\newtheorem{corollary}{Corollary}[theorem] %A environment called corollary is created, the counter of this new environment will be reset every time a new theorem environment is used.
\newtheorem{lemma}[theorem]{Lemma} %In this case, the even though a new environment called lemma is created, it will use the same counter as the theorem environment.

\theoremstyle{definition}
\newtheorem{definition}{Definition}
\newtheorem{example}{Example}[definition]
\newtheorem{remark}{Remark}

\title{Exchangeable Array and Random Graph}
\date{\today}
\author{Dongwoo Kim\\ANU}

\begin{document}

\maketitle

\section{Exchangeable array for graph}
\begin{definition}[Jointly exchangeable array] A random 2-array $(X_{ij})_{i,j\in \mathbb{N}}$
is jointly exchangeable if
\begin{align}
(X_{ij}) \stackrel{d}{=} (X_{\pi(i)\pi(j)}) \quad \text{for } i,j \in \mathbb{N}^2
\end{align}
for any permutation $\pi$ of $\mathbb{N}$.
\end{definition}

\begin{theorem}[\label{aldous_hoover}Aldous, Hoover Theorem \cite{aldous1981representations,hoover1979relations}] A random 2-array $(X_{ij})_{i,j\in \mathbb{N}}$
is jointly exchangeable if and only if there exists a random measurable function $f:[0,1]^3 \rightarrow \mathcal{X}$ such that
\begin{align}
(X_{ij}) \stackrel{d}{=} (f(U_i, U_j, U_{ij})),
\end{align}
where $(U_{i})_{i \in \mathbb{N}}$ and $(U_{ij})_{ij\in\mathbb{N}^2}$ with $U_{ij} = U_{ij}$ are a sequence and matrix of i.i.d. Uniform[0,1] random variables.
\end{theorem}
If the function $f$ is a symmetric in its first two argument, then the matrix $X$ is symmetric. In general, however, it does not need to be a symmetric (e.g. directed graph).

For the undirected graph case $\mathcal{X} = \{0,1\}$, the theorem can be simplified further through a random function called \textit{graphon} $W:[0,1]^2 \rightarrow [0,1]$, symmetric in its arguments, where
\[ f(U_i, U_j, U_{ij}) = 
  \begin{cases}
    1       & \quad U_{ij} < W(U_i, U_j)\\
    0  & \quad \mathrm{otherwise}\\
  \end{cases}
\]

\begin{example}[Random function prior on function $f$ \cite{Lloyd2013}] Lloyd et al. use a Gaussian process prior to define function $f$ for an undirected graph. They define $W(U_i, U_j) = \phi(\Theta(U_i,U_j))$ where $\phi$ is a logstic function, and $\Theta(\cdot, \cdot)$ is a continuous function with a Gaussian process prior. Thus, the probability of edge between node $i$ and $j$ is equal to Bernoulli$(\phi(\Theta(U_i,U_j)))$.
\end{example}

\begin{definition}[Sparse Graph]
Let the number of nodes in a graph be $n$. The graph is sparse if the number of edges are $o(n^2)$ or dense if the number of edges are $\Theta(n^2)$.
\end{definition}

\begin{remark}[Graphon is trivially dense]
Every graph represented by graphon $W$ are either empty or dense. The asymptotic proportion of edges is $p = \frac{1}{2}\int W(x, y) dxdy$ and the graph is hence either empty $(p=0)$ or dense (since $O(pn^2) = O(n^2)$).
\end{remark}

In \cite{Lloyd2013}, the authors place a Gaussian process prior over graphon $W:[0,1]^2 \rightarrow \mathbb{R}$ and transform the output through the logistic function to model the edge probability between nodes.

For the undirected graph case where $X_{ij} = X_{ji}$, one can sample upper triangle of the adjacency matrix, and use the same result for the lower triangle. For the directed graph case, however, the adjacency matrix is no longer symmetric, also, by the theorem, both $X_{ij}$ and $X_{ji}$ rely on single parameter $U_{ij}$. Therefore, one should jointly sample ($X_{ij}$, $X_{ji}$) together from three parameters $U_{i}, U_{j}, U_{ij}$, which means $X_{ij}$ and $X_{ji}$ are not conditionally independent. In addition, asymmetric graphon $W$ could be employed for directed random graph. However, \cite{Cai2015} show that assymetric graphon is inappropriate to impose certain structures on a graph such as the partial ordering and propose a class of priors for directed graphs.

\section{Exchangeable array for matrix factorisation}
Jointly exchangeable array assume that both rows and columns index the same entities. However, the assumption does not hold for the general matrix where rows and columns represent different entities (e.g. users and items in recommendation system).

\begin{definition}[Separately exchangeable array]A random 2-array $(X_{ij})_{i,j\in \mathbb{N}}$ is separately exchangeable if
\begin{align}
(X_{ij}) \stackrel{d}{=} (X_{\pi(i)\sigma(j)}) \quad \text{for } i,j \in \mathbb{N}^2
\end{align}
for any permutation $\pi$ and $\sigma$ of $\mathbb{N}$.
\end{definition}

\begin{corollary}[\label{sea}Separately exchangeable array] A random 2-array $(X_{ij})$ is separately exchangeable if and only if there exists a random measurable function $f:[0,1]^3 \rightarrow \mathcal{X}$ such that
\begin{align}
(X_{ij})  \stackrel{d}{=} (f(U_i, U_j, U_{ij})),
\end{align}
where $(U_i)_{i\in \mathbb{N}}$, $(U_j)_{j\in \mathbb{N}}$, and $(U_{ij})_{ij\in \mathbb{N}^2}$ are i.i.d. Uniform[0,1] random variables.
\end{corollary}
In the separately exchangeable array case, function $f$ is not a symmetric in its first two arguments.

\begin{example}[Probabilistic matrix factorisation (PMF) \cite{Salakhutdinov2008}] 
PMF is one instantiation of Corollary \ref{sea}. Let the generative process of PMF be
\begin{align}
U_i &\sim \mathcal{MN}_d(0, \Sigma_U)\\
V_j &\sim \mathcal{MN}_d(0, \Sigma_V)\\
X_{ij} &\sim \mathcal{N}(U_i^\top V_j, \sigma_x)
\end{align}
where $\mathcal{MN}_d$ is a $d$-dimensional zero-mean multivariate normal distribution with covariance $\Sigma$. This corresponds to random measurable function
$f(U_i, U_j, U_{ij}) = \Phi_1(U_{ij}; \Phi_d(U_i;0, \Sigma_{U})^\top \Phi_d(U_j;0, \Sigma_{U}), \sigma^2_x)$, where $\Phi_d(\cdot;\mu, \sigma^2)$ is an inverse-CDF of $d$-dimensional multivariate function with mean $\mu$ and variance $\sigma^2$. Note that $\sigma$ here differs from the permutation notation.
\end{example}

\section{Exchangeable array for knowledge graph}

A knowledge graph can be represented as a 3-array tensor $X_{ijk}$, where the first two dimensions represent entities, and the third dimension represent relations. Therefore, $X_{ijk}$ is jointly exchangeable in its first two dimensions, and the third dimension is separately exchangeable to the first two dimensions.

\begin{definition}[$\pi$-exchangeable array for knowledge graph \cite{Orbanz2015}]A random 3-array $(X_{ijk})_{i,j,k\in \mathbb{N}}$ is jointly exchangeable in its first two dimensions and separately exchangeable with the third dimension if
\begin{align}
(X_{ijk}) \stackrel{d}{=} (X_{\pi(i)\pi(j)\sigma(k)}) \quad \text{for } i,j,k \in \mathbb{N}^3
\end{align}
for any permutation $\pi$ and $\sigma$ of $\mathbb{N}$.
\end{definition}


\begin{corollary}[$\pi$-exchangeable array for knowledge graph]
A random 3-array $X_{ijk}$ is jointly exchangeable in its first two dimension and separately exchangeable for the third dimension if and only if there exists a random measurable function $f:[0,1]^7 \rightarrow \mathcal{X}$ such that
\begin{align}
(X_{ijk})  \stackrel{d}{=} (f(U_i, U_j, U_k, U_{ij}, U_{ik}, U_{jk}, U_{ijk})),
\end{align}
where $(U_i)_{i\in \mathbb{N}}$, $(U_k)_{k\in \mathbb{N}}$, $(U_{ij})_{ij\in \mathbb{N}^2}$, $(U_{ik})_{ik\in \mathbb{N}^2}$, and $(U_{ijk})_{ijk\in \mathbb{N}^3}$  are i.i.d. Uniform[0,1] random variables.
\end{corollary}

\begin{remark}[Compositional model \cite{gu2015traversing}] Compositional vector space models cannot be modelled through the exchangeable array theorem because the dependency between triples breaks the exchangeability.
\end{remark}

\begin{example}[Bayesian RESCAL (BRESCAL)] The generative process of BRESCAL with normally distributed output variable $X_{ijk}$ is as follows:
\begin{align}
e_i, e_j &\sim \mathcal{MN}_d(0, \sigma_e^2 I_d)\\
R_k &\sim \mathcal{MN}_{d^2}(0, \sigma_r^2 I_{d^2})\\
X_{ijk} &\sim \mathcal{N}(R_k^\top (e_i \otimes e_j), \sigma_x^2).
\end{align}
Again, we can transform uniform random variables $U_i, U_j, U_k$ to $e_i, e_j, R_k$ via the inverse-CDF of zero-mean multivariate normal distribution. Then $X_{ijk}$ is determined by $U_{ijk}$ via the inverse-CDF of normal distribution with mean $R_k^\top (e_i \otimes e_j)$ and variance $\sigma_x^2$, i.e. $X_{ijk} = \Phi_1(U_{ijk};R_k^\top (e_i \otimes e_j), \sigma_x^2)$. The model is invariant w.r.t the changes in the other uniform variables $U_{ij}, U_{ik}$, and $U_{jk}$.
\end{example}

\section{Sparse exchangeable array}
\begin{theorem}[Kallenberg's exchangeable theorem \cite{Kallenberg1990}] Representation theorems for jointly exchangeable random measures on $\mathbb{R}^2$
\end{theorem}

Matrix factorisation for the sparse bipartite graph has been partially discussed in \cite{Caron2012}.

\bibliographystyle{apalike}
\bibliography{ref}

\end{document}
