%!TEX root = ./icml2016.tex

\section{Introduction}

As the amount of information codified in a computer readable fashion increases, the management
of knowledge bases need to become increasingly more automated. In this paper, we study
the problem of acquiring new knowledge, given an existing knowledge base. Knowledge bases
are modeled as a set of relations between pairs of entities, for example the factoid
``Barack Obama is the 44th president of the United States''
is modeled as two entities (Barack Obama, United States) being related by ``president of''.
Such relations have a natural representation as a sparse graph or a tensor of order 3.
Given this representation, we model the acquisition of new knowledge as the
identification of new triples that capture particular relations between two entities.

There are several challenges when we apply machine learning methods to completing
existing knowledge bases, namely:
sparse, noisy, and incomplete annotations.
There has been recent success in transferring ideas from matrix completion problems to
the tensor domain to overcome the challenge of sparsity~\cite{unknown}.
We follow this thread of research by using a low rank approximation model for tensor
factorisation. Such low rank approximations can also be seen as a latent variable probabilistic
model, which additionally captures the inherent uncertainty of noisy annotations.
We propose a probabilistic model for tensor factorisation and explore both the Gaussian
and Logistic model. The probabilistic model provides a natural way of implementing
randomized probability matching, also known as Thompson sampling~\cite{scott10bandit}.
Thompson sampling is an approach for solving the multi-armed bandit problem,
which allows us to trade off exploration and exploitation when identifying new triples.
This provides a principled approach to identify promising candidates for knowledge base
completion.
We additionally consider compositional relations as an additional source of weak information
to further utilise the existing (incomplete) knowledge items.

Goal of this paper 1: Populating knowledge graph with an active label acquisition process corresponding to [B,C,A] in Figure \ref{fig:related3d}.

[B,C,P] could be an alternative direction (or both).
