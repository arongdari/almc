%!TEX root = ./icml2016.tex

\section{Introduction}

As the amount of information codified in a computer readable fashion increases, the management
of knowledge bases need to become increasingly more automated. In this paper, we study
the problem of acquiring new knowledge, given an existing knowledge base. Knowledge bases
are modeled as a set of relations between pairs of entities, for example the factoid
``Barack Obama is the 44th president of the United States''
is modeled as two entities (Barack Obama, United States) being related by ``president of''.
Such relations have a natural representation as a sparse graph or a tensor of order 3.
Given this representation, we model the acquisition of new knowledge as the
identification of new triples that capture particular relations between two entities.

There are several challenges when we apply machine learning methods to completing
existing knowledge bases, namely:
sparse, noisy, and incomplete annotations.
There has been recent success in transferring ideas from matrix completion problems to
the tensor domain to overcome the challenge of sparsity~\cite{unknown}.
We follow this thread of research by using a low rank approximation model for tensor
factorisation. Such low rank approximations can also be seen as a latent variable probabilistic
model, which additionally captures the inherent uncertainty of noisy annotations.
We propose a probabilistic model for tensor factorisation and explore both the Gaussian
and Logistic model. The probabilistic model provides a natural way of implementing
randomized probability matching, also known as Thompson sampling~\cite{scott10bandit}.
Thompson sampling is an approach for solving the multi-armed bandit problem,
which allows us to trade off exploration and exploitation when identifying new triples.
This provides a principled approach to identify promising candidates for knowledge base
completion.
We additionally consider compositional relations as an additional source of weak information
to further utilise the existing (incomplete) knowledge items.

Goal of this paper 1: Populating knowledge graph with an active label acquisition process corresponding to [B,C,A] in Figure \ref{fig:related3d}.

[B,C,P] could be an alternative direction (or both).

\TODO{TODO}{integrate these background discussions and citations}
\rev{
Statistical relational models have been proposed to tackle
various problems of knowledge bases. For example, it can be used 
for question understanding and answering problems \cite{Dong2015},
or it can be integrated into a search engine
to improve the users' search experiment \cite{dong2014knowledge}.

However, most of previous latent feature models lack the consideration
of constructing a knowledge graph with the latent features.
As an alternative approach, distant supervision algorithms
\cite{Mintz2009} suggest a way to fill a missing part of knowledge graph
through the extensive NLP processing, however, this method inherently
requires a set of initial observations used as a distant supervision.

Here, we tackle the problem of knowledge base construction with the
statistical relational model.
A goal of knowledge base construction is to acquire a maximum number
of valid triples with a limited budget.
This goal corresponds to a general objective of the multi-armed bandit (MAP)
problem where we want to minimise the cumulative regret or maximise
the cumulative reward over time. 
In recent years, Thompson sampling has been emerged as  
an competitive solution in MAP problems.
We borrow Thompson sampling, which provides a principled
way to find an optimal trade off between exploration and exploitation,
to solve the knowledge base construction problem.
}

\rev{
We propose a compositional relation model that exploit the compositional structure of 
knowledge graph to capture the latent semantic structure of the entities and relations.
While previously suggested vector space models provide a statistical way to infer the latent semantic 
structure of entities and relations, but lack consideration of a graph structure of a knowledge base itself.

In a separate way from the vector space models, graph feature algorithms such as the path ranking algorithm 
are suggested to fill a missing part of a knowledge graph \cite{Lao2010}. The graph feature algorithms 
directly include graph structures, such as edge type, node type, and node degree, to learn and predict new 
triples, however, the absence of latent structure makes the models failed to predict a new triple when the 
target entities does not have rich structural background\cite{nickel2015review}.

We propose a compositional vector space model that benefits the latent representation of vector space model 
along with the graph structure of the graph feature models. Recently, Guu et. al. suggest a compositional 
training framework for vector space models \cite{gu2015traversing}, where paths over a knowledge graph act 
as a new form of structural regularisation of the models. Based on their work, we extend the compositional 
approach within a probabilistic framework with two compositional structures.
}
