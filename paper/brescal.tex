%!TEX root = icml2016.tex
\section{Bayesian RESCAL}
A relational knowledge base consists of a set triples in the form of $(i, k, j)$
where $i$, $j$ are entities, and $k$ is a relation. A triple can be distinguished
in a valid triple and invalid triple based on a semantic meaning of a triple. An
example of valid triple in Freebase is (Barack Obama, president of, U.S.), and an
example of invalid triple is (Barack Obama, president of, U.K.).
A knowledge base can be represented in a three-way tensor
$\mathcal{X} \in \{0, 1\}^{N \times K \times N}$, where $K$ is a number of
relations, $N$ is a number of entities, and $x_{ikj}\in\mathcal{X}$ indicates whether
the triple is valid.

We model the entities $i$ as vectors $e_i$ and the relations $k$ as matrices $R_k$ with an
appropriately chosen latent dimension $D$. This follows a popular model
for statistical relational learning, which is to factorise the tensor into a
set of latent vector representations, such as the bilinear model RESCAL~\cite{nickel2011three}.
RESCAL aims to factorise each relational slice $X_{:k:}$ into a set of rank-$D$ latent
features as follows:
\[
  \mathcal{X}_{:k:} \approx E R_k E^\top, \qquad \text{for } k = 1, \dots, K
\]
Here, $E\in {\mathbb R}^{N \times D}$ contains the latent features of the
entities $e_1, \ldots, e_N$ and $R_k\in {\mathbb R}^{D \times D}$ models the interaction of the
latent features between entities in relation $k$.

We propose a probabilistic framework that directly generalises RESCAL
by placing priors over the
latent features. For each entity $i$, the latent feature of an entity $e_i \in
\mathbb{R}^{D}$ is drawn from an isotropic multivariate-normal distribution.
\begin{align}
\label{eqn:entity_gen}
e_i \sim {N}(\mathbf{0}, \sigma_e^2{I}_D)
\end{align}
For each relation $k$, we draw matrix $R_k$ from
a zero-mean isotropic matrix normal distribution.
\begin{align}
\label{eqn:relation_gen}
R_k \sim \mathcal{MN}_{D \times D}(\mathbf{0}, \sigma_r{I}_D, \sigma_r{I}_D) \\
\text{or equivalently}\enspace r_k  = \text{vec}(R_k) \sim N(\mathbf{0}, \sigma_r^2 I_{D^2}) \notag
\end{align} %%LX: pull r_k to the front so that it's easy to find
Finally,\eat{we have an observable variable $x_{ikj}$ for each triple in the
knowledge base, We model this variable in two different ways as follows:}
\rev{there are two reasonable distributions for the observable variable $x_{ikj}$.}

\textbf{Logistic output \eat{variable}}: The variable $x_{ikj}$ is a
binary \eat{indicator} variable \rev{indicating whether} the triple is valid or not.
\eat{One natural choice to model the binary variable is to
place a logistic regression model.}
\rev{One natural way to model this is a binomial with its probability determined by logistic regression.}
\begin{align}
p(x_{ikj}=1) = \sigma(e_i^{\top} R_k e_j),
\end{align}
where $\sigma$ is a sigmoid function.

\textbf{Gaussian output \eat{variable}}:
\eat{Second, we}\rev{We can also} place a normal distribution over $x_{ikj}$.
\begin{align}
x_{ikj} |e_i, e_j, R_k \sim \mathcal{N}(e_i^{\top} R_k e_j, \sigma_x^2) = \mathcal{N}(r_k^{\top} e_i \otimes e_j, \sigma_x^2)  \notag %\label{eqn:triple_gen}
\end{align} %%LX: moved this into one line, hopefully not breaking crossrefs
\eat{Although this is not a trivial choice, through the variance parameter $
\sigma_x^2$, we can control the confidence about certain observations. }
\rev{Note that we can control the confidence on different observations}
through the variance parameter $\sigma_x^2$.
\eat{We will discuss more about}
The role of \rev{this parameter will be further discussed} in the compositional model section.

\eat{Given a modelling choice, computing the posterior of latent features $p(E, R|
\mathcal{X})$ is generally intractable. Here, we provide two conditional
posteriors to develop an efficient Gibbs sampler.
} %%LX: flip the sequence/emphasis of the narrative
\rev{We develop an efficient Gibbs sampler to perform inference for Bayesian RESCAL.
The key for achieving efficiency are the two conditional posteriors for latent features.
%This is made possible by two conditional posteriors of latent features,
%as $p(E, R|\mathcal{X})$ is generally intractable.
}
\rev{The posteriors for Gaussian output are listed in Table~\ref{tab:brescalposterior}.}
For \eat{the }logistic regression output, we approximate the conditional posterior of
$E$ and $R$ by Laplace approximation \cite{bishop2006pattern}. The maximum a
posterior estimate of $e_i$ or $R_k$ given the rest can be computed through the
standard logistic regression solvers with regularisation parameters. Given the
maximum a posterior parameters $e_i^*$, the posterior covariance $S_i$ of entity
$i$ takes the form
\begin{align}
S_i^{-1} = \sum_{x_{ikj}} \sigma(e_{i}^{*\top} R_k e_{j}) (1 - \sigma(e_{i}^{*\top} R_k e_{j})) R_k
e_{j}(R_k e_{j})^\top \notag \\
 + \sum_{x_{jki}} \sigma(e_{j}^{\top} R_k e_{i}^*) ( 1- \sigma(e_{j}^{\top} R_k e_{i}^*)) R_k^\top e^*_{i}(R_k^\top + I\sigma_e^{-1}
e^*_{i})^\top. \notag
\end{align}
The posterior covariance of $R_k$ can be computed in the same way.

\begin{table*}[tb]
\caption{Conditional posteriors for Gaussian output. Negative subscript $-i$ indicates the every other entity variables except $e_i$, $\otimes$ is Kronecker product.}\label{tab:brescalposterior}
\begin{tabu}{l}
$p(e_i |E_{-i}, \mathcal{R}, \mathcal{X}^{t}, \sigma_e, \sigma_x) = \mathcal{N}(e_i | \mu_i, \Lambda_i^{-1})$ \\
with
$~\mu_i = \frac{1}{\sigma_x^2}\Lambda_i^{-1}\xi_i$,
$\Lambda_i = \frac{1}{\sigma_x^2} \sum_{jk : x_{ikj} \in \mathcal{X}^{t}} (R_k e_j)(R_k e_j)^\top$,
$\xi_i = \sum_{jk : x_{ikj} \in \mathcal{X}^{t}}  x_{ikj} R_{k} e_{j} +
\sum_{jk : x_{jki} \in \mathcal{X}^{t}} x_{jki} R_{k}^\top e_{j}.$
\\ \hline
$p(R_k|E, \mathcal{X}, \sigma_r, \sigma_x)  = \mathcal{N}(\text{vec}(R_k) |
\mu_k, \Lambda_k^{-1})$ \\
with
$\mu_k = \frac{1}{\sigma_x^2}\Lambda_k^{-1}\xi_k$,
$\Lambda_k = \frac{1}{\sigma_x^2} \sum_{ij:x_{ikj} \in \mathcal{X}^{t}} (e_i
\otimes e_j)(e_i \otimes e_j)^\top + \frac{1}{\sigma_r^2} {I}_{D^2}$,
$\xi_k = \sum_{ij:x_{ikj} \in \mathcal{X}^{t}} x_{ikj} (e_{i} \otimes e_{j}).$
\\ \hline
$p(x_{ikj}| E, \mathcal{X}^{t}, \sigma_x, \sigma_r)
= \mathcal{N}(x_{ikj}| \mu_k ^\top (e_i \otimes e_j), \frac{1}{\sigma_x^2} +
(e_i \otimes e_j)^\top \Lambda_k (e_i \otimes e_j))$
\end{tabu}
\end{table*}

\eat{%LX: these are all in the table now.
For the gaussian output case, the conditional posterior of $e_i$ given $E_{-i}$
and $R$ or $R_k$ given $E$ are straight forward. We use the negative subscript $-
i$ to indicate the every other entity variables except $e_i$. The
conditional distribution of $e_i$ given $\mathcal{R}$ and other entities $E_{-i}$
is
\begin{align} \label{eqn:sample_e}
p(e_i |E_{-i}, \mathcal{R}, \mathcal{X}^{t}, \sigma_e, \sigma_x) &= \mathcal{N}
(e_i | \mu_i, \Lambda_i^{-1}),
\end{align}
where
\begin{align*}
\mu_i &= \frac{1}{\sigma_x^2}\Lambda_i^{-1}\xi_i \\
\Lambda_i &= \frac{1}{\sigma_x^2} \sum_{jk : x_{ikj} \in \mathcal{X}^{t}} (R_k
e_j)(R_k e_j)^\top \\
&\quad+ \frac{1}{\sigma_x^2} \sum_{jk : x_{jki} \in \mathcal{X}^{t}} (R_k^\top
e_j)(R_k^\top e_j)^\top+ \frac{1}{\sigma_e^2} {I}_D \\
\xi_i &= \sum_{jk : x_{ikj} \in \mathcal{X}^{t}}  x_{ikj} R_{k} e_{j} +
\sum_{jk : x_{jki} \in \mathcal{X}^{t}} x_{jki} R_{k}^\top e_{j}.
\end{align*}
Let $\otimes$ be a Kronecker product. The conditional distribution of $R_k$ given
$E$ is
\begin{align}
\label{eqn:sample_r}
p(R_k|E, \mathcal{X}, \sigma_r, \sigma_x)  &= \mathcal{N}(\text{vec}(R_k) |
\mu_k, \Lambda_k^{-1}),
\end{align}
where
\begin{align*}
\mu_k &= \frac{1}{\sigma_x^2}\Lambda_k^{-1}\xi_k \\
\Lambda_k &= \frac{1}{\sigma_x^2} \sum_{ij:x_{ikj} \in \mathcal{X}^{t}} (e_i
\otimes e_j)(e_i \otimes e_j)^\top + \frac{1}{\sigma_r^2} {I}_{D^2} \\
\xi_k &= \sum_{ij:x_{ikj} \in \mathcal{X}^{t}} x_{ikj} (e_{i} \otimes e_{j}).
\end{align*}
With the gaussian output, the posterior marginal predictive distribution of
$x_{ikj}$ given $\mathcal{X}$ and $E$ is
\begin{align}
\label{eqn:marginal_predict}
&p(x_{ikj}| E, \mathcal{X}^{t}, \sigma_x, \sigma_r) \\
&= \mathcal{N}(x_{ikj}| \mu_k ^\top (e_i \otimes e_j), \frac{1}{\sigma_x^2} +
(e_i \otimes e_j)^\top \Lambda_k (e_i \otimes e_j)). \notag
\end{align}
}
