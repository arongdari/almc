%!TEX root = icml2016.tex
\section{Bayesian RESCAL}
\label{sec:brescal}

\begin{table*}[bt]
\caption{Parameters for Gibbs updates. $\otimes$ is the Kronecker product.}
\label{tab:brescalposterior}
\vskip 0.05in
\begin{tabu}{l|l|l}
$\mu$&$\Lambda$&$\xi$\\
\hline
$\frac{1}{\sigma_x^2}\Lambda_i^{-1}\xi_i$&
$\frac{1}{\sigma_x^2} \sum_{jk : x_{ikj} \in \mathcal{X}^{t}} (R_k e_j)(R_k e_j)^\top$&
$\sum_{jk : x_{ikj} \in \mathcal{X}^{t}}  x_{ikj} R_{k} e_{j} +
\sum_{jk : x_{jki} \in \mathcal{X}^{t}} x_{jki} R_{k}^\top e_{j}.$
\\
$\frac{1}{\sigma_x^2}\Lambda_k^{-1}\xi_k$&
$\frac{1}{\sigma_x^2} \sum_{ij:x_{ikj} \in \mathcal{X}^{t}} (e_i
\otimes e_j)(e_i \otimes e_j)^\top + \frac{1}{\sigma_r^2} {I}_{D^2}$&
$\sum_{ij:x_{ikj} \in \mathcal{X}^{t}} x_{ikj} (e_{i} \otimes e_{j}).$
\end{tabu}
\end{table*}


A relational knowledge base consists of a set triples in the form of $(i, k, j)$
where $i$, $j$ are entities, and $k$ is a relation. A triple can be distinguished
in a valid triple and invalid triple based on a semantic meaning of a triple. An
example of valid triple in Freebase is (BarackObama, PresidentOf, U.S.), and an
example of invalid triple is (BarackObama, PresidentOf, U.K.).
A knowledge base can be represented in a three-way tensor
$\mathcal{X} \in \{0, 1\}^{N \times K \times N}$, where $K$ is a number of
relations, $N$ is a number of entities, and $x_{ikj}\in\mathcal{X}$ indicates whether
the triple is valid.

We model the entities $i$ as vectors $e_i$ and the relations $k$ as matrices $R_k$ with an
appropriately chosen latent dimension $D$. This follows a popular model
for statistical relational learning, which is to factorise the tensor into a
set of latent vector representations, such as the bilinear model RESCAL~\cite{nickel2011three}.
RESCAL aims to factorise each relational slice $X_{:k:}$ into a set of rank-$D$ latent
features as follows:
\[
  \mathcal{X}_{:k:} \approx E R_k E^\top, \qquad \text{for } k = 1, \dots, K
\]
Here, $E\in {\mathbb R}^{N \times D}$ contains the latent features of the
entities $e_1, \ldots, e_N$ and $R_k\in {\mathbb R}^{D \times D}$ models the interaction of the
latent features between entities in relation $k$.

We propose a probabilistic framework that directly generalises RESCAL
by placing priors over the
latent features. For each entity $i$, the latent feature of an entity $e_i \in
\mathbb{R}^{D}$ is drawn from an isotropic multivariate-normal distribution.
\begin{align}
\label{eqn:entity_gen}
e_i \sim {N}(\mathbf{0}, \sigma_e^2{I}_D)
\end{align}
For each relation $k$, we draw matrix $R_k$ from
a zero-mean isotropic matrix normal distribution.
\begin{align}
\label{eqn:relation_gen}
R_k \sim \mathcal{MN}_{D \times D}(\mathbf{0}, \sigma_r{I}_D, \sigma_r{I}_D) \\
\text{or equivalently}\enspace r_k  = \text{vec}(R_k) \sim N(\mathbf{0}, \sigma_r^2 I_{D^2}) \notag
\end{align}
where $\text{vec}(R_k)$ denotes the flattening of the matrix.

We consider two models for $x_{ikj}$: a real or random variable. By placing a
normal distribution over $x_{ikj}$,
\begin{align}
  x_{ikj} |e_i, e_j, R_k \sim \mathcal{N}(e_i^{\top} R_k e_j, \sigma_x^2) \label{eqn:triple_gen}
\end{align}
we can control the confidence on different observations
through the variance parameter $\sigma_x^2$.
The role of this parameter will be further discussed in the compositional model section.

We develop an efficient Gibbs sampler to perform inference for Bayesian RESCAL (BRESCAL).
The key for achieving efficiency are the two conditional posteriors for latent features.
The Gibbs updates are given by:
\begin{align}
p(e_i |E_{-i}, \mathcal{R}, \mathcal{X}^{t}, \sigma_e, \sigma_x) &= \mathcal{N}(e_i | \mu_i,
\Lambda_i^{-1})  \label{eqn:sample_e} \\
p(R_k|E, \mathcal{X}, \sigma_r, \sigma_x) &= \mathcal{N}(\text{vec}(R_k) |
\mu_k, \Lambda_k^{-1}) \label{eqn:sample_r}
\end{align}
where the negative subscript $-i$ indicates the every other entity variables except $e_i$.
The means and precision matrices are listed in Table~\ref{tab:brescalposterior}, where we have
used the identity $e_i^{\top} R_k e_j = r_k^{\top} e_i \otimes e_j$.

Alternatively, we may want to more closely model the fact that the observations are binary.
Therefore we model $x_{ikj}$ as a binomial distributed random variable whose
probability is determined by logistic regression.
\[
p(x_{ikj}=1) = \sigma(e_i^{\top} R_k e_j),
\]
where $\sigma$ is a sigmoid function.
We approximate the conditional posterior of
$E$ and $R$ by Laplace approximation \cite{bishop2006pattern}. The maximum a
posterior estimate of $e_i$ or $R_k$ given the rest can be computed through the
standard logistic regression solvers with regularisation parameters. Given the
maximum a posteriori parameters $e_i^*$, the posterior covariance $S_i$ of entity
$i$ takes the form
\begin{align*}
S_i^{-1} = \sum_{x_{ikj}} \sigma(e_{i}^{*\top} R_k e_{j}) (1 - \sigma(e_{i}^{*\top} R_k e_{j})) R_k
e_{j}(R_k e_{j})^\top\\
 + \sum_{x_{jki}} \sigma(e_{j}^{\top} R_k e_{i}^*) ( 1- \sigma(e_{j}^{\top} R_k e_{i}^*)) R_k^\top e^*_{i}(R_k^\top e^*_{i})^\top + I\sigma_e^{-1}
.
\end{align*}
The posterior covariance of $R_k$ can be computed in the same way, and is shown in the appendix.

There are many advantages to a Bayesian view of tensor factorisation, such as
the quantification of uncertainty by the predictive distribution,
the ability to utilise priors, and
the availability of principled model selection.
We show in the empirical experiments that BRESCAL outperforms standard RESCAL.
In the following, we focus on the predictive distribution, which enables us to
improve sequential knowledge acquisition.
