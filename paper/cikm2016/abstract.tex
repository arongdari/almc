%!TEX root = ./cikm2016.tex
\begin{abstract}
\eat{
%Automated processes for knowledge base completion benefit from models that account for
%statistical uncertainty. 
We are concerned with the problem of knowledge base completion, or inferring missing facts from known relations. 
%  It is desirable to take into account paths and compositions, and 
%  able to connect to the knowledge extraction problem by actively 
%  acquiring labeled data points. 
%  Recent knowledge completion algorithms could do either, but not both. 
A human can readily infer a new fact through the composition of known facts in knowledge base,
whereas the current statistical relational models lack the consideration of
active knowledge acquisition through the composition of knowledge.
We propose a probabilistic model to bridge this gap. 
We start from a new formulation of vector space embedding for knowledge tensors  %which generalises the RESCAL model to 
that explicitly model distributions of relations. This
enables us to extend Thompson sampling approach to knowledge bases, and 
%we demonstrate the benefit of active knowledge acquisition. We 
also incorporate additive and multiplicative
approaches for composing relations. 
On synthetic and real world datasets, 
we find that learning with composition is helpful 
when training data is sparse,
and that Thompson sampling provides effective exploitation-exploration strategies 
that balance recall and reconstruction accuracy. 
% and show the regimes where compositional models are
%beneficial for knowledge base completion.
}

A knowledge base construction consists of a two-step approach; extracting information from external sources, known as knowledge extraction, and inferring missing information through a statistical analysis on the extracted information, known as knowledge completion.
In many cases, however, it is implausible to extract a fair amount of information from the external sources. An active knowledge acquisition via labelling of human experts can help to reduce the gap between two processes. 
In this paper, we propose a new probabilistic knowledge base factorisation that benefits from a compositionality of existing knowledge (e.g. syllogism). This explicit probabilistic formulation enable us to develop an active acquisition model based on exploitation-exploration strategies.
We demonstrate that the compositional knowledge factorisation results a better performance on the knowledge completion, whereas the model performs worse in the active knowledge acquisition.
The result leads to a counter-intuitive conclusion; a better predictive model does not guarantee to have a better active acquisition model.
An additional experiment explains the degeneracy in terms of the exploitation-exploration regime in the active knowledge acquisition.

\end{abstract}
