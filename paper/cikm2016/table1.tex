%!TEX root = ./cikm2016.tex

\begin{table}[t]
\centering
\caption{\label{tbl:relatedwork}The categorisation of factorisation problems with respect to 
three design considerations. The column headings are Probabilistic(Pr)/Non-Probabilistic(N-Pr) method, Passive(P)/Active(A) learning, and Matrix(M)/Tensor(T)/Compositional(C) structure. In this work, we tackle the problems denoted by an asterisk.}
\vskip 0.15in
\begin{tabular}{c c c l}
Pr/N-Pr & P/A & M/T/C & References	\\ \hline \hline

N-Pr & P & M & \cite{lee1999learning}\\ \hline
N-Pr & A & M & \cite{ruchansky2015matrix}\\  \hline

\multirow{2}{*}{N-Pr} & \multirow{2}{*}{P} & \multirow{2}{*}{T}& \cite{nickel2011three}\\
& & & \cite{kolda2009tensor}\\ \hline
N-Pr & A & T & \cite{kajino2015active} \\  \hline

\multirow{2}{*}{N-Pr} & \multirow{2}{*}{P} & \multirow{2}{*}{C} & \cite{Neelakantan2015} \\ 
& & & \cite{guu2015traversing}\\ \hline

N-Pr & A & C & -- \\ \hline

Pr & P & M & \cite{mnih2007probabilistic}\\ \hline

\multirow{2}{*}{Pr} & \multirow{2}{*}{A} & \multirow{2}{*}{M}&  \cite{kawale2015efficient} \\
& & & \cite{sutherland2013active}\\ \hline

\multirow{2}{*}{Pr} & \multirow{2}{*}{P} & \multirow{2}{*}{T}& *, \cite{xiong2010temporal}\\
& & & \cite{schmidt2009probabilistic} \\ \hline

Pr & A & T & * \\ \hline

Pr & P & C & * \\ \hline

Pr & A & C & * \\ 
\end{tabular}
\end{table}