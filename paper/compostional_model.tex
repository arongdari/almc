%!TEX root = ./icml2016.tex
\section{Compositional Relations}
In this section, we propose a compositional relation model that exploit the compositional structure of knowledge graph to capture the latent semantic structure of the entities and relations. 
While previously suggested vector space models provide a statistical way to infer the latent semantic structure of entities and relations, but lack consideration of a graph structure of a knowledge base itself.

In a separate way from the vector space models, graph feature algorithms such as the path ranking algorithm are suggested to fill a missing part of a knowledge graph \cite{Lao2010}. The graph feature algorithms directly include graph structures, such as edge type, node type, and node degree, to learn and predict new triples, however, the absence of latent structure makes the models failed to predict a new triple when the target entities does not have rich structural background\cite{nickel2015review}.

We propose a compositional vector space model that benefits the latent representation of vector space model along with the graph structure of the graph feature models. Recently, Guu et. al. suggest a compositional training framework for vector space models \cite{gu2015traversing}, where paths over a knowlede graph act as a new form of structural regularisation of the models. Based on their work, we extend the compositional approach within a probabilistic framework with two compositional structures.

The compositionality represents a semantic meaning of a path over a knowledge graph that corresponds to a sequence of composable triples. 
For example, given two triples, ``Barack Obama is a 44th president of U.S.'' (Barack Obama / president of / U.S) and ``Joe Biden was a running mate of Barack Obama'' (Joe Biden / running mate of / Barack Obama), one can naturally deduce that the ``Joe Biden is a vice president of U.S.'' (Joe Biden / vice president of / U.S.). Here the composition of two relations, president of, and running mate of, yield to a compositional relation, vice president of.
% Is every running mate always a vice president? maybe this is not a proper example.
More formally, if there is a sequence of triples where the target entity of a former triple is a source entity of a latter triple in a consecutive pair of triples in the sequence, then we can form a compositional triples as follows.
Given the sequence of triples
$(i_1, k_1 ,j_1)$,  $(i_2, k_2, j_2)$, $(i_2, k_2, j_2)$ $\dots$ $(i_n, k_n, j_n)$, where $i_k = j_{k+1}$ for all $k$,  we form a compositional triple $(i_1, {c}(k_1, k_2, \dots, k_n), j_n)$, where $c$ denotes the compositional relation of the sequence of relations.

Let $\mathcal{C}^{L}$ be a set of all possible compositions of which length is up to $L$, $c \in \mathcal{C}$ be a sequence of relations, $c(i)$ be $i$th index of a relation in sequence $c$ and $|c|$ be the length of the sequence. With set of compositions $\mathcal{C}^{L}$, we can expand set of observed triples $\mathcal{X}^{t}$ to set of compositional triples $\mathcal{X}^{\mathcal{C}^{L}(t)}$ in which compositional triple $x_{icj}$ is an indicator variable that show the existence of the path from entity $i$ to entity $j$ through sequence of relations $c$ in $\mathcal{X}^{t}$. Note that the compositional relation $c$ is an abstract relation, and there might be a multiple possible paths from $i_1$ to $j_n$.

With these extended compositional triples, we again model $x_{icj}$ with a bilinear Gaussian distribution,
\begin{align}
x_{(i, {{c}(k_1, k_2)}, l)} \sim \mathcal{N}(e_i^\top R_{{c}(k_1,k_2)} e_j, \sigma_{c}^2),
\end{align}
where $R_{{c}(k_1,k_2)} \in \mathbb{R}^{D\times D}$ is a latent matrix of compositional relation $c$, and $\sigma_{c}^2$ is a covariance of the compositional triples. We keep the same latent vector $e$ for each entity to model both normal triples and compositional triples.
In the subsequent sections, we provide two different ways of modelling the compositional relation $R_c$.

\subsection{Additive Compositionality}
First, we define an additive compositional relation $R_c$ as a sequence of normalized summation over relation matrices in composition $c$, i.e.,
$R_{{c}} = \frac{1}{|c|}(R_{c(1)} + R_{c(2)} + \dots + R_{c(|c|)})$, then compositional triple $x_{icj}$ is modeled as
\begin{align}
x_{(i, c, j)} &\sim \mathcal{N}(e_i^\top R_c e_j, \sigma_{c}^2) \\
&= \mathcal{N}(e_i^\top \frac{1}{|c|}(R_{c(1)} + R_{c(2)} + \dots + R_{c(|c|)}) e_j, \sigma_{c}^2). \notag
\end{align}
The conditional distribution of $e_i$ given $E_{-i}, \mathcal{R}, \mathcal{X}^{t}, \mathcal{X}^{L(t)}$ is  expanded from the posterior of BRESCAL by incorporating compositional triples.
\begin{align} \label{eqn:comp_sample_e}
p(e_i |E_{-i}, \mathcal{R}, \mathcal{X}^{t}, \mathcal{X}^{L(t)}) &= \mathcal{N}(e_i | \mu_i, \Lambda_i^{-1}),
\end{align}
where
\begin{align*}
\mu_i &= \Lambda_i^{-1}\xi_i \\
\Lambda_i &= \frac{1}{\sigma_x^2} \sum_{jk : x_{ikj} \in \mathcal{X}^{t}} (R_k e_j)(R_k e_j)^\top \\
&\quad+ \frac{1}{\sigma_x^2} \sum_{jk : x_{jki} \in \mathcal{X}^{t}} (R_k^\top e_j)(R_k^\top e_j)^\top \\
&\quad + \frac{1}{\sigma_c^2} \sum_{jc : x_{icj} \in \mathcal{X}^{L(t)}} (R_c e_j)(R_c e_j)^\top \\
&\quad+ \frac{1}{\sigma_c^2} \sum_{jc : x_{jci} \in \mathcal{X}^{L(t)}} (R_c^\top e_j)(R_c^\top e_j)^\top + \frac{1}{\sigma_e^2} {I}_D \\
\xi_i &= \frac{1}{\sigma_x^2}\sum_{jk : x_{ikj} \in \mathcal{X}^{t}}  x_{ikj} R_{k} e_{j} + \frac{1}{\sigma_x^2}\sum_{jk : x_{jki} \in \mathcal{X}^{t}} x_{jki} R_{k}^\top e_{j} \\
& + \frac{1}{\sigma_c^2}\sum_{jc : x_{icj} \in \mathcal{X}^{L(t)}}  x_{icj} R_{c} e_{j} + \frac{1}{\sigma_c^2}\sum_{jc : x_{jci} \in \mathcal{X}^{L(t)}} x_{jci} R_{c}^\top e_{j}
\end{align*}

To compute the conditional distribution of $R_k$, we first decompose $R_c$ into two part where $R_c = \frac{1}{|c|} R_k + \frac{|c|-1}{|c|}R_{c/k}$, where $R_{c/k} = \sum_{k' \in c/k} R_{k'}$. 
The distribution of compositional triple is decomposed as follows:
\begin{align}
x_{(i, c, l)} \sim \mathcal{N}(e_i^\top (\frac{1}{|c|} R_k + \frac{|c|-1}{|c|}R_{c/k}) e_j, \sigma_{c}^2).
\end{align}
Then, the conditional distribution $R_k$ given $R_{-k}, E, \mathcal{X}^{t}, \mathcal{X}^{L(t)}$ is
\begin{align}
\label{eqn:comp_cond_r}
p(R_k|E, \mathcal{X}^{t}, \mathcal{X}^{L(t)}, \sigma_r, \sigma_x)  &= \mathcal{N}(\text{vec}(R_k) | \mu_k, \Lambda_k^{-1}),
\end{align}
where
\begin{align*}
\mu_k &=\Lambda_k^{-1}\xi_k \\
\Lambda_k &= \frac{1}{\sigma_x^2} \sum_{ij:x_{ikj} \in \mathcal{X}^{t}} \bar{e}_{ij}\bar{e}_{ij}^\top + \frac{1}{\sigma_r^2} {I}_{D^2} \\
& +\frac{1}{|c|^2 \sigma_c^2} \sum_{ij:x_{icj} \in \mathcal{X}^{L(t)},\text{ }k \in c} \bar{e}_{ij} \bar{e}_{ij}^\top \\
\xi_k &=  \frac{1}{\sigma_x^2}\sum_{ij:x_{ikj} \in \mathcal{X}^{t}} x_{ikj} \bar{e}_{ij}\\
& +\frac{1}{|c| \sigma_c^2} \sum_{ij:x_{icj} \in \mathcal{X}^{L(t)},\text{ }k \in c} x_{icj} \bar{e}_{ij} - \frac{|c|-1}{|c|} \bar{e}_{ij} r_{c/k}^\top \bar{e}_{ij}\\
\bar{e}_{ij} &= e_{i} \otimes e_{j}.
\end{align*}
Vectorisation of $R_c$ and $R_{c/k}$ are represented as $r_c$ and $r_{c/k}$, respectively.

%Note that the ordering of the relations in compositional sequence $c$ does not affect the value of compositional triple $(i, c, j)$.

\subsection{Multiplicative Compositionality}
Second, we define an multiplicative compositional relation $R_c$ as a sequence of multiplication over relations in composition $c$, i.e. $R_c = R_{c(1)} R_{c(2)} \dots R_{c(|c|)}$, and the compositional triple as a bilinear Gaussian distribution with the compositional relation $R_c$,
\begin{align}
x_{(i, c, j)} \sim \mathcal{N}(e_i^\top R_{c(1)}R_{c(2)} \dots R_{c(|c|-1)}R_{c(|c|)} e_j, \sigma_{c}^2)
\end{align}
The multiplicative compositionality can be understood as a sequence of linear transformation from the original entity $i$ with the compositional relations, and the inner product between the transformed entity and target entity will form a value of the compositional triple.

Given a sequence of relations including relation $k$, $R_k$ is placed in the middle of the compositional sequence, i.e., $e_i^\top R_{c(1)}R_{c(2)} \dots R_{c(\delta_k)} \dots R_{c(|c|-1)}R_{c(|c|)} e_j$, where $\delta_k$ is the index of relation $k$. For notational simplicity, we will denote the left side $e_i^\top R_{c(1)}R_{c(2)} \dots R_{c(\delta_k -1)}$ as $\bar{e}_{ic(:\delta_k)}^\top$, and the right side $R_{c(\delta_k + 1)} \dots R_{c(|c|-1)}R_{c(|c|)} e_j$ as $\bar{e}_{ic(\delta_k:)}$, therefore we can rewrite the mean parameter as $\bar{e}_{ic(:\delta_k)}^\top R_{k} \bar{e}_{ic(\delta_k:)}$. With the simplified notations, the conditional of $R_k$ is
\begin{align}
p(R_k|E, \mathcal{X}, \sigma_r, \sigma_x)  &= \mathcal{N}(\text{vec}(R_k) | \mu_k, \Lambda_k^{-1}),
\end{align}
where
\begin{align*}
\mu_k &= \Lambda_k^{-1}\xi_k \\
\Lambda_k &= \frac{1}{\sigma_x^2} \sum_{ij:x_{ikj} \in \mathcal{X}^{t}} (e_i \otimes e_j)(e_i \otimes e_j)^\top + \frac{1}{\sigma_r^2} {I}_{D^2} \\
+ &\frac{1}{\sigma_c^2} \sum_{ij:x_{icj} \in \mathcal{X}^{L(t)}, \text{ }k \in c} (\bar{e}_{ic(:\delta_k)} \otimes \bar{e}_{jc(\delta_k:)})(\bar{e}_{ic(:\delta_k)} \otimes \bar{e}_{jc(\delta_k:)} )^\top \\
\xi_k &= \frac{1}{\sigma_x^2} \sum_{ij:x_{ikj} \in \mathcal{X}^{t}} x_{ikj} (e_{j} \otimes e_{i}) \\
& + \frac{1}{\sigma_c^2} \sum_{ij:x_{icj} \in \mathcal{X}^{L(t)}, \text{ }k\in c} x_{icj} (\bar{e}_{ic(:\delta_k)}  \otimes \bar{e}_{jc(\delta_k:)}).
\end{align*}
The conditional distribution of $e_i$ given the rest is the same as Equation $\ref{eqn:sample_e}$. 

As the length of sequence $c$ increases, a small error in the first few multiplication will result a large differences in the final compositional relation. One way to mitigate the cascading error is to increase variance of compositional triples $\sigma_c$ as the length of the sequence increases.

%If the determinant of latent relation $R_k$ is greater than 1, the compositional latent relation $R_c$ might be exploded after multiplying a long sequence of relations. To obtain a stable scale of compositional relation $R_c$, one may multiply decaying factor $\tau < 1$ after each composition. $R_c = \tau^{|c|-1} R_{c(1)} R_{c(2)} \dots R_{c(|c|)}$.

