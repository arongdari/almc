%!TEX root = icml2016.tex

\begin{table}[t]
\centering
\caption{\label{tbl:relatedwork}The categorisation of factorisation problems with respect to 
three design considerations. The column headings are Bayesian(B)/Non-Bayesian(N) method, Passive(P)/Active(A) learning, and Matrix(M)/Tensor(T)/Compositional(C) structure. In this work, we tackle the problems denoted by an asterisk.}
\vskip 0.15in
\begin{tabular}{c c c l}
B/N & P/A & M/T/C & References	\\ \hline \hline

N & P & M & \citet{lee1999learning}\\ \hline
N & A & M & \citet{ruchansky2015matrix}\\  \hline

\multirow{2}{*}{N} & \multirow{2}{*}{P} & \multirow{2}{*}{T}& \citet{nickel2011three}\\
& & & \citet{kolda2009tensor}\\ \hline
N & A & T & \citet{kajino2015active} \\  \hline

\multirow{2}{*}{N} & \multirow{2}{*}{P} & \multirow{2}{*}{C} & \citet{Neelakantan2015} \\ 
& & & \citet{guu2015traversing}\\ \hline

N & A & C & -- \\ \hline

B & P & M & \citet{mnih2007probabilistic}\\ \hline

\multirow{2}{*}{B} & \multirow{2}{*}{A} & \multirow{2}{*}{M}&  \citet{kawale2015efficient} \\
& & & \citet{sutherland2013active}\\ \hline

\multirow{2}{*}{B} & \multirow{2}{*}{P} & \multirow{2}{*}{T}& *, \citet{xiong2010temporal}\\
& & & \citet{schmidt2009probabilistic} \\ \hline

B & A & T & * \\ \hline

B & P & C & * \\ \hline

B & A & C & * \\ 
\end{tabular}
\end{table}